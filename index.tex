\documentclass[]{article}

%opening
\title{How Open is Open Source? Metrics for measuring software mutation}
\author{Charles Hathaway}

\begin{document}

\maketitle

\begin{abstract}

Open source software is toted as being "openly accessible" to many people, thus allowing greater innovation and complex new methodologies.
However, given the complexity of software and how difficult it is to design and implement, the question of whether or not secondary communities adapt and modify the software needs to be addressed.
This paper will first summarize previous works regarding the structure of open source communities, discuss how useful or not useful previous attempts to quantify the "open-ness" of projects have been, and finally propose a metric for measuring how open a project is.
We will conclude with a proposal for a technique to test the proposed metric.

\end{abstract}

\section{Introduction}

% What is open source? very brief regurgitation
% Why do we care about open source? list large projects that "matter"
% --> Include projects listed as "can not fail" internet services
% Does it matter how open projects are?
% How will being able to measure how open projects are help organize things

\section{Literature Review}

% We have at least 2 distinct "areas" to review; open source-ness, and ways of measuring software similiarity

% Search for previous works on the following keywords:
% --> Open source project topologies
% --> Measuring software "changes"

% More stuff?

\section{Discussion and comparison of previous metrics}

\section{A new metric}

\section{Experiment proposals}

% Ron's proposal goes in here;
% As a remind for the me that forgets later;
% --> Design a base CSnap project
% --> Modify the project to target a specific metric
% ---> This will be done multiple times; 1 project for each metric, then combinations
% --> Ask "people" which projects represent the most change, the least
% --> See which metrics agree with the responses the most

\section{Conclusion}
%% Moorthy
\section{References}
%% Please add papers in bibtex format
%%

\end{document}
