\documentclass[]{article}

\usepackage[backend=bibtex]{biblatex}

%opening
\title{How Open is Open Source? Metrics for measuring software mutation}
\author{Charles Hathaway}

\bibliography{biblo}

\begin{document}

\maketitle

\begin{abstract}

Open source software is toted as being "openly accessible" to many people, thus allowing greater innovation and complex new methodologies.
However, given the complexity of software and how difficult it is to design and implement, the question of whether or not secondary communities adapt and modify the software needs to be addressed.
This paper will first summarize previous works regarding the structure of open source communities, discuss how useful or not useful previous attempts to quantify the "open-ness" of projects have been, and finally propose a metric for measuring how open a project is.
We will conclude with a proposal for a technique to test the proposed metric.

\end{abstract}

\section{Introduction}

% What is open source? very brief regurgitation
% Why do we care about open source? list large projects that "matter"
% --> Include projects listed as "can not fail" internet services
% Does it matter how open projects are?
% How will being able to measure how open projects are help organize things

\section{Literature Review}

% We have at least 2 distinct "areas" to review; open source-ness, and ways of measuring software similiarity

% Search for previous works on the following keywords:
% --> Open source project topologies
% --> Measuring software "changes"

%%%% Open source project topologies %%%%
% Initial google search returned links to GEOS, JTS Topology Suite
% Search on scholar.google.com more fruitful; link to http://dl.acm.org/citation.cfm?id=381535

\subsection{A case study of the evolution of Jun: an object-oriented open-source 3D multimedia library}

% Once I get this into the biblo, change to \ref or whatever
This paper discusses the development of an open source graphics library that focuses on ease of use over performance.
Some notable things to consider; the project was written in Smalltalk (the same language as the original Scratch), it still exists, and it was primarily developed by a small team (rather than a community)
% -> Company site: http://www.cincomsmalltalk.com/main/community/product-portal/contributed/jun/
% -> Source code: http://aokilab.kyoto-su.ac.jp/jun/index.html
% This poses the interesting question of whether or not development teams are usually small, or large?
% -> Asked in ref:two_case_studies

%%%%% analysis of open source network %%%%

% First response is The open source software development phenomenon: An analysis based on social network theory
\ref{ref:open_source_network_theory} uses graph theory to map developers to projects, and create a "graph" which they discuss in the form of network theory.
This is very similiar to what I had hoped to do with Github, and the graph is fascinating.
They coin the term "linchpin developer" to talk about developers who tie together projects.
Very interesting paper, but it needs better formatting...
% -> This led to a great find of papers by searching for papers which reference this one
% --> http://scholar.google.com/scholar?cites=3750653309408772417&as_sdt=5,33&sciodt=0,33&hl=en





% More stuff?

\section{Discussion and comparison of previous metrics}

% We need to talk about everything, including silly things like...
% Lines of code changed

\subsection{Cyclomatic complexity}

\cite{ref:a_complexity_measure}

\subsection{Normalized Compression Distance}

% Maybe?
\cite{ref:cilibrasi2005clustering}

\subsection{Effort measure}

% Need to trace this back I think
% This is sort of the thing that Ron's expirement would look at
\cite{ref:evaluating_software_complexity_measures}

\subsection{Data flow complexity}

\cite{ref:oviedo1993control}

\subsection{Complexity Measure Based on Program Slicing (CMBPS)}

% This is an aggregate complexity measure
% Very recent work!
\cite{ref:tao2014complexity}

\section{A new metric}

% We really need to decide what to do about this :(

% Propose new metric in this paper and test with the toy example (small)

% Most likely an aggregate metric, but what weights and metrics?

\section{Experiment proposals}

% Ron's proposal goes in here;
% As a remind for the me that forgets later;
% --> Design a base CSnap project
% --> Modify the project to target a specific metric
% ---> This will be done multiple times; 1 project for each metric, then combinations
% --> Ask "people" which projects represent the most change, the least
% --> See which metrics agree with the responses the most

% Repeat with other groups; programmers, crowd sourcing

\section{Conclusion}
%% Moorthy
\section{References}
%% Please add papers in bibtex format
%%

\printbibliography

\end{document}
